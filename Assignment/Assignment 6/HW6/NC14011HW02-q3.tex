\subsection*{الف}

تعداد پارامترهای لازم در این بخش برابر است با پارامترهای بردار اولیه به علاوه‌ی پارامترهای مورد نیاز برای ماتریس انتقال و همچنین نشان دادن احتمال مربوط به مشاهدات:

$$
k^2 + k + km	
$$

\subsection*{ب}

الگوریتم
\lr{Forward}
یکی از الگوریتم‌های مهم در حوزه بیزین
\lr{(Bayesian)}
است و برای تخمین احتمالات پسین
\lr{(posterior probabilities)}
متغیرهای نهان در مدل‌های گرافیکی بیزین استفاده می‌شود.

این الگوریتم برای محاسبه احتمالات پسین متغیرهای نهان با استفاده از داده‌های مشاهده شده به کار می‌رود. به طور کلی، الگوریتم
\lr{Forward}
بر اساس فرآیند تکراری اطلاعات جدید را درباره احتمالات پسین بروز می‌دهد.

برای شروع، احتمال پیشین
\lr{(prior probability)}
متغیرهای نهان را تعیین می‌کنیم. سپس با استفاده از این احتمال پیشین و اطلاعات مشاهده شده، احتمالات پیشین تا زمان فعلی را محاسبه می‌کنیم.

با توجه به الگوریتم
\lr{forward}
و توالی اتفاقات، احتمال هر کدام را می‌نویسیم.

$$
P(A, O_1) = 0.8 \times 0.99 = 0.792
$$

$$
P(B, O_1) = 0.1 \times 0.01 = 0.001
$$

$$
P(A, O_{1 | 2}) = 0.2 \times 0.78418 = 0.158
$$

$$
P(B, O_{1 | 2}) = 0.9 \times 0.008019 = 0.0072	
$$

$$
P(A, O_{1} | 3) = 0.8 \times 0.78418
$$

$$
P(B, O_{1 | 3}) = 0.1 \times 0.00949
$$

\subsection*{ج}

الگوریتم
\lr{Backward}
یکی از روش‌های استفاده شده در الگوریتم‌های الگوی مخفی مارکوف


\lr{(Hidden Markov Models)}
است. این الگوریتم برای تخمین متغیرهای نهان مدل مارکوف مخفی بر اساس داده‌های مشاهده شده استفاده می‌شود.

در
\lr{Backward algorithm}
، ما به دنبال محاسبه احتمال مشاهده شدن تمام دنباله‌ها تا یک زمان مشخص در آینده هستیم. برای این کار، از پس‌بینی
\lr{(backtracking)}
برای حساب کردن این احتمالات استفاده می‌شود.

برای شروع، ما از زمان پایانی به زمان شروع عقب می‌رویم و در هر مرحله محاسبات خاصی را انجام می‌دهیم. احتمال پایانی را در زمان پایانی به عنوان شرط اولیه قرار می‌دهیم و سپس احتمال تمام دنباله‌ها تا زمان قبلی را محاسبه می‌کنیم.

در این بخش باید احتمال‌ها را برعکس و از جلو به عقب محاسبه کنیم، پس داریم:

$$
P_3(A) = 0.8 \times 0.99 + 0.001 = 
$$

$$
P_3(B) = 0.8 \times 0.01 + 0.1 \times 0.99 = 0.107
$$

$$
P_2(A) = 0.793 \times 0.198 + 0.107 \times 0.009 = 0.158
$$

$$
P_2(B) = 0.793 \times 0.002 + 0.107 \times 0.891 = 0.095
$$

$$
P_1(A) = 0.158 \times 0.792 + 0.095 \times 0.001
$$

$$
P_1(B) = 0.158 \times 0.008 + 0.095 \times 0.099
$$

\subsection*{د}

همان 
\lr{forward}
است که به جای جمع، ماکسیمم می‌گیریم پس داریم:

$$
P(A, O_1) = 0.8 \times 0.99 = 0.792
$$

$$
P(B, O_1) = 0.1 \times 0.01 = 0.001
$$

$$
P(A, O_{1 | 2}) = 0.2 \times max(0.78408 , 0.0001) = 0.158
$$

$$
P(B, O_{1 | 2}) = 0.9 \times max(0.00792 , 0.00099) = 0.007	
$$

$$
P(A, O_{1} | 3) = 0.8 \times max(0.15543 , 0.00008)
$$

$$
P(B, O_{1 | 3}) = 0.1 \times max(0.00157 , 0.00792)
$$

