به محاسبه‌ی نقاط مرزی تابع که بپردازیم می‌بینیم که مقادیر
$f(0)$
و
$f(4)$
برابر با ۰ هستند.
حالا مقادیر
$f(1)$
و
$f(2)$
و
$f(3)$
را محاسبه می‌کنیم با توجه به این‌که هر سطر احتمال رفتن به دیگر حالت‌ها را می‌دهد.

$$
f(1) = p(1 + 0) + q(1 + f(2)) = p + q + qf(2) = 1 + qf(2)
$$

$$
f(2) = p(1 + f(1)) + q(1 + f(3)) = p + q + pf(1) + qf(3) = 1 + pf(1) + qf(3)
$$

$$
f(3) = p(1 + f(2)) + q(1 + 0) = p + q + pf(2) = 1 + pf(2)
$$

پس بدست می‌آوریم:

$$
f(1) = 1 + \frac{2q}{q - 2pq}
$$

$$
f(2) = \frac{2}{1 - 2pq}
$$

$$
f(3) = 1 + \frac{2p}{1 - 2pq}
$$

پس در نهایت مقدار هر کدام می‌شود:

$$
f(1) = 3
$$

$$
f(2) = 4
$$

$$
f(3) = 1
$$

و داریم:

$$
f(n) = 4n - n^2
$$

