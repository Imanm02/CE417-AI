در اینجا از آزمون t ولچ برای مقایسه دو گروه استفاده می‌کنیم. فرمول آزمون t ولچ به صورت زیر است:
\begin{gather*}
	t = \frac{\bar{X_1} - \bar{X_2}}{\sqrt{\frac{\sigma_1^2}{N_1} + \frac{\sigma_2^2}{N_2}}}
\end{gather*}
در اینجا، $\bar{X_1}$ و $\bar{X_2}$ به ترتیب میانگین دو گروه را نشان می‌دهند. همچنین $\sigma_1^2$ و $\sigma_2^2$ واریانس دو گروه و $N_1$ و $N_2$ تعداد نمونه‌ها در هر گروه را نشان می‌دهند.

با استفاده از این فرمول و با داشتن مقادیر میانگین‌ها ($X_1 = 44$ و $X_2 = 57$) و واریانس‌ها ($\sigma_1^2 = 82.5$ و $\sigma_2^2 = 154.3$) و تعداد نمونه‌ها ($N_1 = 5$ و $N_2 = 7$)، مقدار t را محاسبه می‌کنیم:
\begin{gather*}
	t = \frac{57 - 44}{\sqrt{\frac{82.5}{5} + \frac{154.3}{7}}} = 2.094
\end{gather*}

همچنین در اینجا درجه آزادی
\lr{(degrees of freedom)}
را نیز محاسبه می‌کنیم که به صورت زیر است:
\begin{gather*}
	df = N_1 + N_2 - 2 = 7 + 5 - 2 = 10
\end{gather*}

سپس برای محاسبه‌ی مقدار
\lr{p-value}
از زبان
\lr{R}
و پکیج
\lr{p-test}
استفاده می‌کنیم. برای محاسبه
\lr{p-value}
، از تابع
\lr{pt}
در
\lr{R}
استفاده می‌کنیم. مقدار
\lr{p-value}
برابر است با دو برابر احتمالی که
\lr{t}
استخراج شده از توزیع
\lr{t}
با درجه آزادی
\lr{df}
بزرگتر یا مساوی
\lr{t}
محاسبه شده باشد:
\begin{latin}
	p-value=2*pt(2.094, df = 10, lower=FALSE)≈0.063p-value =
\end{latin}

\begin{latin}
2*pt(2.094, df = 10, lower = FALSE)≈0.063
\end{latin}

سپس با مقایسه
\lr{p-value}
با سطح اهمیت
$\alpha = 0.05$
، اگر
\lr{p-value}
بیشتر از 
$\alpha$
باشد، نتیجه می‌گیریم که نتایج آماری معنادار نیست و تفاوت آماری معناداری وجود ندارد. در اینجا، زیرا
$0.063 > 0.05$ 
است، پس می‌توان نتیجه گرفت که تفاوت بین دو گروه معنادار نیست.