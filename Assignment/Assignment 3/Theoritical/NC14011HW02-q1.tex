\subsection*{الف}

با یک متغیر تصادفی به نام $S$ کار می‌کنیم. تابع توزیع تجمعی (CDF) این متغیر تصادفی با $F_S(s)$ نمایش داده می‌شود. سپس می‌خواهیم $F_S(s)$ را به شکلی معادل با دیگر متغیرهای تصادفی توضیح دهیم.

$F_S(s)$
را می‌توان به صورت زیر نوشت:
\begin{equation*}
	F_S(s) =
	\begin{cases}
		s & \text{} 0 \le s < 1 \\
		2 - s & \text{ } 1 \le s \le 2 \\
		0 & \text{در غیر این صورت}
	\end{cases}
\end{equation*}

حال می‌توانیم به توضیح روابطی که در متن برقرار است، بپردازیم. ابتدا با استفاده از تعریف تابع توزیع تجمعی، معادله زیر را می‌توان نوشت:
\begin{equation*}
	F_S(s) = P(S < s)
\end{equation*}

سپس با استفاده از اعلانی که در متن داده شده، $S$ را می‌توان به صورت جمع دو متغیر تصادفی $X$ و $Y$ نمایش داد:
\begin{equation*}
	S = X + Y
\end{equation*}

حال با جایگذاری معادله $S = X + Y$ در $P(S < s)$ و استفاده از قاعده توزیع شرطی، می‌توان نوشت:
\begin{equation*}
	P(S < s) = P(X + Y < s)
\end{equation*}

سپس، با استفاده از تعریف تابع توزیع تجمعی متغیر تصادفی $X$، $P(X < s - Y)$ را می‌توان به صورت زیر بازنویسی کرد:
\begin{equation*}
	P(X + Y < s) = P(X < s - Y)
\end{equation*}

با در نظر گرفتن اینکه مقدار $Y$ ثابت است، می‌توانیم از رابطه $\mathbb{E}(X) = \mathbb{E}(\mathbb{E}(X \mid Y))$ استفاده کنیم و معادله بالا را به شکل زیر بازنویسی کنیم:
\begin{equation*}
	P(X < s - Y) = \mathbb{E}(P(X < s - Y \mid Y = y))
\end{equation*}

با استفاده از قاعده بازنویسی توزیع شرطی، می‌توانیم $P(X < s - Y \mid Y = y)$ را به شکل $F_X(s - y)$ بازنویسی کنیم. پس معادله بالا به شکل زیر ساده می‌شود:
\begin{equation*}
	\mathbb{E}(P(X < s - Y \mid Y = y)) = \mathbb{E}(F_X(s - y))
\end{equation*}

در اینجا، به دلیل محدود بودن بازه مقداردهی به $s$، یعنی $0 \le s \le 2$، می‌توانیم معادله بالا را به صورت یک انتگرال بر حسب $y$ نوشت:
\begin{equation*}
	\mathbb{E}(F_X(s - y)) = \int_{-\infty}^\infty F_X(s - y)f_Y(y) dy
\end{equation*}

دقت کنید که $f_X(s-y)$ و $f_Y(y)$ تابع چگالی احتمال (PDF) متغیرهای تصادفی $X$ و $Y$ را نشان می‌دهند.

حال با توجه به بازه‌های مقداردهی به $s$، می‌توانیم به دو حالت متفاوت معادله بالا را مورد بررسی قرار دهیم. این دو حالت عبارتند از: $0 \le s \le 1$ و $1 \le s \le 2$.

حال، برای هر یک از این دو حالت، انتگرال را بر حسب $y$ حل می‌کنیم.

در حالت اول، وقتی $0 \le s \le 1$، با توجه به بازه‌های مقداردهی به $y$ و $s - y$، می‌توانیم انتگرال را به شکل زیر بنویسیم:
\begin{equation*}
	\int_{-\infty}^\infty F_X(s - y)f_Y(y) dy = \int_0^s 1 dy = s
\end{equation*}

حال، به حالت دوم، یعنی $1 \le s \le 2$، می‌پردازیم. در این حالت نیز، با توجه به بازه‌های مقداردهی به $y$ و $s - y$، انتگرال به صورت زیر حل می‌شود:
\begin{equation*}
	\int_{-\infty}^\infty F_X(s - y)f_Y(y) dy = \int_{s - 1}^1 1 dy = 2 - s
\end{equation*}

با توجه به نتایج به دست آمده در دو حالت، می‌توانیم تابع توزیع تجمعی $F_S(s)$ را به شکل زیر نوشت:
\begin{equation*}
	F_S(s) =
	\begin{cases}
		s & \text{برای } 0 \le s < 1 \\
		2 - s & \text{برای } 1 \le s \le 2 \\
		0 & \text{در غیر این صورت}
	\end{cases}
\end{equation*}

با این توضیحات، می‌توانیم روشن‌تر درک کنیم که چگونه متغیرهای تصادفی $X$ و $Y$ منجر به متغیر تصادفی $S$ می‌شوند و چگونه تابع توزیع تجمعی $S$ به دست می‌آید.

\subsection*{ب}

با استفاده از تعریف تابع چگالی احتمال مشروط، تابع چگالی احتمال شرطی 
$f_{X|S}(x | s)$
را به صورت زیر بیان می‌کنیم:
\begin{equation*}
	f_{X|S}(x | s) = \frac{f_{X,S}(x, s)}{f_S(s)} = \frac{f_{X,Y}(x, s - x)}{f_S(s)}
\end{equation*}

حال، با توجه به بازه مقداردهی به $s$ که در متن اصلی ذکر شده است، تابع چگالی احتمال شرطی را در چند حالت متفاوت مورد بررسی قرار می‌دهیم. این حالات عبارتند از:
$0 \le x \le s < 1$، $0 \le x \le 1 \le s < 2$
و سایر حالات.

در حالت اول، وقتی 
$0 \le x \le s < 1$
، با توجه به بازه مقداردهی به $x$ و $s$، تابع چگالی احتمال شرطی به صورت زیر بدست می‌آید:
\begin{equation*}
	f_{X|S}(x | s) =
	\begin{cases}
		\frac{f_{X,Y}(x, s - x)}{s} = \frac{s - x}{s} & \text{برای } 0 \le x \le s < 1
	\end{cases}
\end{equation*}

در این حالت، تابع چگالی احتمال شرطی برابر با نسبت فاصله $s - x$ تا $s$ به $s$ است. یعنی مقدار $X$ به مرور از 0 به $s$ افزایش می‌یابد و با افزایش $x$، احتمال کاهش می‌یابد.

حال، به حالت دوم، یعنی $0 \le x \le 1 \le s < 2$ می‌پردازیم. در این حالت، تابع چگالی احتمال شرطی به صورت زیر محاسبه می‌شود:
\begin{equation*}
	f_{X|S}(x | s) =
	\begin{cases}
		\frac{f_{X,Y}(x, s - x)}{2 - s} = 1 & \text{برای } 0 \le x \le 1 \le s < 2, s - x < 1\\
		\frac{f_{X,Y}(x, s - x)}{2 - s} = 0 & \text{برای } 0 \le x \le 1 \le s < 2, s - x > 1\\
	\end{cases}
\end{equation*}

در این حالت، مقدار تابع چگالی احتمال شرطی برابر 1 است زیرا $X$ و $Y$ به طور مستقل از هم توزیع شده‌اند و مقدار $X$ در بازه $[0,1]$ تعریف شده است. بنابراین، برای همه مقادیر $x$ در این بازه وقتی که $s - x < 1$ باشد، احتمال برابر 1 است و در غیر این صورت، احتمال برابر صفر است.

در سایر حالات، تابع چگالی احتمال شرطی برابر با صفر است.

\subsection*{ج}

برای محاسبه امید ریاضی $X$ به شرط $S = 0.5$، از تابع چگالی احتمال شرطی استفاده می‌کنیم. طبق قسمت قبل، در این حالت، تابع چگالی احتمال شرطی برابر است با $\frac{s - x}{s}$.

با جایگذاری $s = 0.5$ و محاسبه امید ریاضی، داریم:
\begin{gather*}
	\mathbb{E}(X | S = 0.5) = \int_{0}^{0.5} x \frac{0.5 - x}{0.5} dx
\end{gather*}

حال می‌توانیم با محاسبه این انتگرال، مقدار امید ریاضی $X$ به شرط $S = 0.5$ را تقریباً محاسبه کنیم. این مقدار تقریباً برابر با $0.01667$ است.

\subsection*{د}

برای محاسبه امید ریاضی $M$، ما نیاز داریم که به تمام حالت‌های محتمل که در بخش دوم بدست آورده‌ایم، امید ریاضی را محاسبه کنیم.

طبق بخش ب، حالاتی که برای محاسبه امید ریاضی در نظر گرفته می‌شوند عبارتند از:
\begin{gather*}
	\mathbb{E}(M) =
	\begin{cases}
		\int_{0}^{s} \frac{s - x}{s} dx & \text{برای } 0 \le x \le s < 1\\
		\int_{0}^{s} 1 dx = s & \text{برای } 0 \le x \le 1 \le s < 2, s - x < 1\\
		0 & \text{در غیر این صورت}\
	\end{cases}
\end{gather*}

حال، با استفاده از فرمول انتگرالی امید ریاضی و با محاسبه این انتگرال‌ها، می‌توانیم مقدار امید ریاضی $M$ را بدست آوریم.	