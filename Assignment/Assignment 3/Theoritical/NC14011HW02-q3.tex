با توجه به اینکه اطلاعات خاصی از توزیع این متغیرها در اختیار نداریم، برای حدس زدن که
\lr{W}
از
\lr{Z}
بیشتر است یا خیر، می‌توانیم استراتژی ساده‌ای را پیش بگیریم: به صورت تصادفی حدس بزنیم.

یعنی در هر بار، با یک احتمال نیمه به نیمه، حدس بزنیم که
\lr{W}
بیشتر از Z است یا خیر. با این روش، احتمال اینکه حدس ما درست باشد حدود
$1/2$
است.

برای اینکه احتمال موفقیت این استراتژی برابر با
$\epsilon + 1/2$
شود، اگر
$\epsilon$
را از مجموعه اعداد حقیقی کوچکتر از
$1/2 $
در نظر بگیریم، می‌توانیم یک ترفند استفاده کنیم: اگر
\lr{W}
و
\lr{Z}
هر دو عدد حقیقی باشند و متفاوت باشند (که به این معنی است که یکی بیشتر و دیگری کمتر است)، اگر فرض کنیم
\lr{W}
بیشتر از
\lr{Z}
است، در
$\epsilon$ 
درصد موارد، ما به جای حدس
$Z<W$
، حدس
$Z>W$
را انتخاب می‌کنیم. این روش موجب می‌شود احتمال موفقیت ما در حدس زدن درست برابر با 
$\epsilon + 1/2$
شود.

به عبارت دیگر، اگر
\lr{X}
یک متغیر تصادفی با توزیع برنولی با پارامتر
$\epsilon$ 
باشد، استراتژی ما به این صورت است که اگر
$X=1$
، ما حدس می‌زنیم
$Z>W$
و اگر
$X=0$
، حدس می‌زنیم
$Z<W$
. از آنجا که
\lr{X}
با احتمال
$\epsilon$ 
برابر با
$1 $
 است، این استراتژی می‌تواند احتمال موفقیت ما را به
$\epsilon + \frac{1}{2}$
 برساند.

توجه داشته باشید که این استراتژی فقط در صورتی کاربرد دارد که
$\epsilon$
کوچکتر از
$\frac{1}{2}$
باشد. اگر
$\epsilon$
بزرگتر از
$\frac{1}{2}$ 
باشد، استراتژی مناسبی وجود ندارد که احتمال موفقیت ما را به
$\epsilon + \frac{1}{2}$
برساند، زیرا در حالت کلی و با داشتن دانسته‌های فعلی، احتمال موفقیت ما هرگز نمی‌تواند بیشتر از
$\frac{1}{2}$
باشد.
 
 
 