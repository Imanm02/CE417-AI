بله، الگوریتم
\lr{Naive Bayes Classifier}
یکی از الگوریتم‌های مهم و پرکاربرد در یادگیری ماشین است. در این الگوریتم، احتمال اینکه یک نمونه به یک دسته خاص از دسته‌های مسئله تعلق داشته باشد، بر اساس احتمالات شرطی و احتمالات آپریوری محاسبه می‌شود.

فرض کنید
$C$
مجموعه‌ای از
$k$
دسته مسئله باشد و 
$X = (x_1, x_2, \dots, x_n)$
برداری از
$n$
ویژگی مربوط به یک نمونه باشد. هدف
\lr{Naive Bayes Classifier}
، 
یافتن دسته‌ای است که بهترین احتمال اینکه نمونه با ویژگی‌های
$X$
 به آن تعلق داشته باشد، را دارد.

برای این کار، ابتدا احتمالات آپریوری 
$P(C)$
برای هر یک از دسته‌ها محاسبه می‌شود. سپس برای هر یک از ویژگی‌ها، احتمال شرطی 
$P(x_i|C)$
برای هر دسته و هر ویژگی محاسبه می‌شود. به عبارت دیگر، ما به دنبال پیدا کردن احتمال این هستیم که ویژگی 
$x_i$
در دسته 
$C$
وجود داشته باشد. این احتمالات با استفاده از داده‌های آموزشی محاسبه می‌شوند.

با توجه به این احتمالات، ما می‌توانیم احتمال اینکه نمونه با ویژگی‌های 
$X$
به دسته 
$C$
تعلق داشته باشد، را برای هر دسته 
$C$
محاسبه کنیم. این احتمال برای هر 
$C$
به شکل زیر است:

$$
P(C|X) = \frac{P(C) \prod_{i=1}^n P(x_i|C)}{P(X)}
$$

در اینجا، 
$P(C|X)$
احتمال شرطی برای دسته 
$C$
به شرط داشتن ویژگی‌های 
$X$
است، 
$P(C)$
احتمال آپریوری دسته 
$C$
است، 
$P(x_i|C)$
احتمال شرطی برای ویژگی 
$x_i$
در دسته 
$C$
است و 
$P(X)$
احتمال ظاهر شدن ویژگی‌های 
$X$
در تمام دسته‌ها است.

احتمال آپریوری 
$P(C)$
و احتمال شرطی 
$P(x_i|C)$
با استفاده از الگوریتم‌های مختلفی محاسبه می‌شوند، از جمله الگوریتم‌های تخمین تک بعدی
\lr{Gaussian}
، تخمین تک بعدی 
\lr{Bernoulli}
و تخمین چند بعدی
\lr{Gaussian}
.
در ضمن، فرض نادرستی بسیار ساده‌ای که در این الگوریتم اعمال می‌شود، فرض نادرستی
\lr{Naive Bayes}
نام دارد. این فرض به این صورت است که تمام ویژگی‌ها مستقل از یکدیگر هستند، به عبارت دیگر، هیچ تداخلی بین آن‌ها وجود ندارد.

در کل، الگوریتم
\lr{Naive Bayes Classifier}
به دلیل سرعت بالا و دقت خوب در بسیاری از مسائل، از الگوریتم‌های محبوب در یادگیری ماشین است. البته، این الگوریتم در مواردی که فرض نادرستی
\lr{Naive Bayes}
قابل قبول نیست، به خوبی عمل نمی‌کند.

\subsection*{الف}


$$
P(Spam | w_1 , w_2 , w_3 , w_4 , w_5) = P(Spam) \prod_{i=1}^5 P(w_i | Spam) = \frac{1}{54} 
$$

$$
P(Not - Spam | w_1 , w_2 , w_3 , w_4 , w_5) = P(Not - Spam) \prod_{i=1}^5 P(w_i | Not - Spam) = \frac{9}{1280} 
$$

در نتیجه احتمالا اسپم است.

\subsection*{ب}

$$
P(Spam | w_1 , w_2 , \neg w_3 , \neg w_4 , \neg w_5) = P(Spam) \prod_{i=1}^2 P(w_i | Spam) \prod_{i=3}^5 P(\neg w_i | Spam) = 0
$$

$$
P(Not - Spam | w_1 , w_2 , \neg w_3 , \neg w_4 , \neg w_5) = 
$$

$$
P(Not - Spam) \prod_{i=1}^2 P(w_i | Not - Spam) \prod_{i=3}^5 P(\neg w_i | Not - Spam) = \frac{1}{1280}
$$

در داده‌های داده شده هیچ نمونه اسپمی نداریم  ولی طبق احتمال‌هاِ، ایمیل اسپم است.

\subsection*{ج}

$$
P(Spam | w_1 , w_2 , \neg w_3 , \neg w_4 , \neg w_5) = P(Spam) \prod_{i=1}^2 P(w_i | Spam) \prod_{i=3}^5 P(\neg w_i | Spam) =
$$

$$
\frac{6+5}{10+10} \times \frac{5+1}{6+5} \times \frac{4+1}{6+5} \times \frac{5+1}{6+5} \times \frac{4+1}{6+5} \times \frac{0+1}{6+5} \approx 0.003
$$



$$
P(Not - Spam | w_1 , w_2 , \neg w_3 , \neg w_4 , \neg w_5) 
$$

$$
= P(Not - Spam) \prod_{i=1}^2 P(w_i | Not - Spam) \prod_{i=3}^5 P(\neg w_i | Not - Spam) =
$$

$$
\frac{4+5}{10+10} \times \frac{1+1}{4+5} \times \frac{1+1}{4+5} \times \frac{1+1}{4+5} \times \frac{1+1}{4+5} \times \frac{2+1}{4+5} \approx 0.0003
$$

پس ایمیل اسپم شناخته می‌شود.
