

افزایش داده‌ی آموزش: این کار می‌تواند مشکل را حل کند، به خصوص اگر مشکل
\lr{overfitting}
 باشد. با داشتن داده‌های آموزش بیشتر، مدل می‌تواند الگوهای عمومی تری را یاد بگیرد و این می‌تواند به جلوگیری از
\lr{overfitting}
کمک کند. اما باید توجه داشت که این روش می‌تواند هزینه‌بر و زمان‌بر باشد و همچنین نمی‌تواند همیشه مشکل
\lr{underfitting}
را حل کند.

کاهش پارامتر regularization : این کار می‌تواند به حل مشکل
\lr{underfitting}
کمک کند. Regularization یک تکنیک است که برای جلوگیری از overfitting استفاده می‌شود، به طوری که با افزایش آن، مدل می‌تواند ساده‌تر شود. اما اگر مدل در حال حاضر underfitting دارد، کاهش regularization می‌تواند به مدل اجازه دهد تا پیچیدگی بیشتری را یاد بگیرد.

افزایش پارامتر regularization : این کار می‌تواند به حل مشکل overfitting کمک کند. 

با افزایش regularization ، مدل می‌تواند ساده‌تر شود و بنابراین احتمال overfitting کاهش می‌یابد. اما اگر مدل در حال حاضر underfitting دارد، افزایش regularization می‌تواند مشکل را بدتر کند.

استخراج ویژگی‌های بهتر از داده: این کار می‌تواند در هر دو حالت، یعنی overfitting و underfitting ، مفید باشد. استخراج ویژگی‌های بهتر می‌تواند به مدل کمک کند تا الگوهای مهم‌تر و عمومی‌تری را در داده‌ها بیابد. این کار می‌تواند به جلوگیری از overfitting کمک کند و همچنین می‌تواند به مدل کمک کند تا با کمک ویژگی‌های مهم‌تر، پیچیدگی‌های موجود در داده‌ها را بهتر یاد بگیرد و بنابراین به حل مشکل underfitting کمک کند.