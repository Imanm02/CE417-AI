\subsection*{الف}

در ابتدا احتمالاتی که در آن‌ها
$X_1$
داریم را باید محاسبه کنیم و
$Elimination$
انجام دهیم.
با توجه به این‌که هر کدام از متغیرها دو حالت دارند، بک جدول (بزرگ‌ترین جدول ممکن) با سایز 
$2^6 = 64$
خواهیم داشت در مرحله‌ای که می‌خواهیم 
$X_1$
را حذف کنیم که برابر
$X_1$
تا
$X_6$
می‌باشد. برای کوچک‌تر کردن سایز جدول هم می‌توان حاصل احتمالاتی که در آن‌ها
$X_2 , X_3 , ... , X_6$
وجود دارند با ضرب و جمع را بدست آوریم و سایز جدول را کوچک‌تر کنیم در نتیجه بزرگ‌ترین سایز جدول ۶۴ می‌باشد. 
در این شبکه‌های بیزین چون حین
$join$
راس
$X_1$
، همه‌ی بچه‌های آن نیز باید جوین شوند، از
$X_2$ 
تا
$X_{n+1}$
، و چون هر کدام دو حالت دارند و تعداد فرزندان 
$X_1$
نیز 
$n$
است، در کل
$2^{n+1}$
حالت داریم.

\subsection*{ب}

در ابتدا 
$X_2$
تا
$X_5$
را حذف می‌کنیم و در نهایت نیز 
$X_1$
را حذف می‌کنیم. از آن‌جایی که بعد از حذف هر کدوم، فقط با 
$X_1$
ارتباط خواهند داشت، پس جدول ما
$2^2 = 4$
حالت خواهد داشت.
\subsection*{ج}

اگر به ازای هر زیردرخت، فرزندها را ابتدا
$join$
کنیم و پس از 
$join$
آخرین فرزند، به سراغ ریشه برویم، در نهایت جدولی با سایز ۴ خواهیم داشت حین عملیات
$join$
چون همه‌ی 
$n-1$
فرزند قبلی، 
$join$
و
$eliminate$
شده‌اند و دیگر در جدول حضور ندارند.

پس هر دو ترتیب زیر جواب‌گو هستند برای خواسته‌ی سوال:

$$
X_5 , X_6 , X_7 , X_8 , X_9 , X_{10} , X_2 , X_3 , X_4 , X_1
$$

$$
X_5 , X_6 , X_2 , X_7 , X_8 , X_3 , X_9 , X_{10} , X_4 , X_1
$$