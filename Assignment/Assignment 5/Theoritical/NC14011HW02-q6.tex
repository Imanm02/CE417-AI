\subsection*{الف}

در این بخش، تعداد متغیرهای وابسته به هر یک از متغیرها را بررسی می‌کنیم:

متغیر A : به B وابسته است پس جدول ۴ حالت دارد.


متغیر B : به متغیری وابسته نیست پس جدول ۲ حالت دارد.


متغیر C : به B وابسته است پس جدول ۴ حالت دارد.


متغیر D : به B وابسته است پس جدول ۴ حالت دارد.


متغیر E : به B و C وابسته است پس جدول ۸ حالت دارد.


متغیر F : به E و C و D وابسته است پس جدول ۱۶ حالت دارد.


در کل به ۶ جدول برای ذخیره‌سازی نیاز است.
\subsection*{ب}

خیر. ممکن است یال اضافی رسم شده باشد. همچنین هر یال نشان‌گر وابستگی دو متغیر به یکدیگر است و می‌توان جهت یال‌ها را نیز عوض کرد و باز هم جواب درست گرفت.

\subsection*{ج}

نمی‌توان شبکه‌ی دیگری کشید که معادل این شبکه باشند.

\subsection*{د}

برای گراف کامل معنای خاصی ندارد حتما. اگر تمامی رئوس به همدیگر وصل باشند، همه‌ی متغیرها به هم وابسته هستند که بی‌معنا است و شبکه نیست. اگر شبکه بدون یال باشد نیز به معنی مستقل بودن همه‌ی متغیرها است. اگر گراف بدون راس باشد نیز به معنی وجود نداشتن متغیر و وجود نداشتن شبکه است.


