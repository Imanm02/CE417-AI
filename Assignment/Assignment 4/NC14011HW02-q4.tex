\subsection*{الف}

وقتی از عبارت «در صورتی که \lr{discount} را نداشته باشیم» استفاده می‌کنیم، منظور از 
\lr{discount}
میزان تاثیر یا اهمیتی است که به مراحل آینده در محاسبه 
\lr{reward}
نسبت داده می‌شود. در الگوریتم‌ها و مدل‌های مختلف یادگیری تقویتی، از این عامل برای تعیین اهمیت مراحل آینده در محاسبه 
\lr{reward}
استفاده می‌شود. با داشتن 
\lr{discount}
، مقدار 
\lr{reward}
 برای مراحلی که در آینده قرار خواهند گرفت، کاهش می‌یابد.

بنابراین، در حالتی که 
\lr{discount}
را نداشته باشیم، ممکن است که به دلیل عدم توجه به اهمیت مراحل آینده، 
\lr{reward}
کلی که در 1000 مرحله بدست می‌آید، بیشتر به نظر برسد و ما بیشتر تمایل داشته باشیم آن را انتخاب کنیم. اما وقتی 
\lr{discount}
وجود دارد، 
\lr{reward}
ی که بعد از 10 مرحله بدست می‌آید ممکن است اهمیت بیشتری داشته باشد و ما تمایل داشته باشیم آن را انتخاب کنیم.

همچنین، ممکن است که در مواقعی که دو حالت به تعداد مراحل یکسانی کمیت مشابهی از 
\lr{reward}
را ارائه می‌دهند، ما متردد شویم و تصمیم‌گیری را به تعویق بیندازیم. این به معنای این است که می‌توانیم در انتخاب بین این دو حالت کمی اضافه زمان صرف کنیم تا مطمئن شویم کدام گزینه بهتر است.

به طور کلی، در صورتی که عواملی مانند 
\lr{discount}
در نظر گرفته نشوند، اهمیت مراحل آینده و تفاوت بین 
\lr{reward}
ها کاهش می‌یابد و این می‌تواند در فرآیند تصمیم‌گیری تاثیر داشته باشد.

\subsection*{ب}

بله، بالاخره این مراحل باید طی شوند و در نهایت به یک جواب همگرا می‌شوند.

\subsection*{ج}

در رابطه‌ی ارائه شده برای جفت حالت‌ها، 
$R_{max|min}$
نشان‌دهنده‌ی حداکثر یا حداقل مقدار
\lr{reward}
است که در آن جفت حالت بدست می‌آید. همچنین،
$\gamma$
مقداری بین 0 و 1 است که به آن عامل تخفیف گفته می‌شود و نشان‌دهنده‌ی اهمیت مراحل آینده در محاسبه‌ی
\lr{reward}
است.

در صورتی که
$\gamma < 1$
باشد، جمع هندسی 
$\sum \gamma^i$
به مقدار 
$\frac{1}{1-\gamma}$
همگرا می‌شود. این به این معنی است که اهمیت مراحل آینده در محاسبه‌ی
\lr{reward}
با افزایش تعداد مراحل کاهش می‌یابد، اما به طور متناسب با عامل تخفیف گفته می‌شود. به عبارت دیگر، هر چه 
$\gamma$
کوچک‌تر باشد، اهمیت مراحل آینده بیشتر خواهد بود.

اگر 
$\gamma \geq 1$
باشد، مجموعه
$\sum \gamma^i$
به نام همگرایی دارای مقدار بی‌نهایت می‌شود. این به این معنی است که در این حالت، اهمیت مراحل آینده بی‌نهایت است و تاثیر مراحل از لحاظ زمانی نامحدود است.

بنابراین، برای اطمینان از همگرایی وجود عامل تخفیف، مقدار 
$\gamma$
باید بین 0 و 1 قرار گیرد. در نهایت، مقدار 
$U_2$
که نشان‌دهنده‌ی
\lr{utility}
حالت است، به 
$\frac{R_{max|min}}{1-\gamma}$
همگرا می‌شود. این به این معنی است که با در نظر گرفتن عامل تخفیف،
\lr{utility}
حالت برابر با تقسیم مقدار حداکثر یا حداقل
\lr{reward}
بر تفاوت یک منهای 
$\gamma$
خواهد بود. این مقدار نشان‌دهنده‌ی ارزش حالت در نظر گرفته شده با اهمیت مراحل آینده است.


